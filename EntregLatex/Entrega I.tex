\documentclass[a4paper,12pt]{article}

\usepackage[catalan]{babel}
\usepackage{fontspec}
\usepackage{fullpage}
\usepackage[a4paper, margin=2cm]{geometry} % To change the margins
\usepackage{graphicx} % Insert images
\usepackage[hidelinks]{hyperref} % Links color
\usepackage{import}
\usepackage{ragged2e}
\usepackage{wrapfig} %To Text wrap
\usepackage{listings} % Add code
\usepackage{tikz}
\usepackage{calc}


%Tikz defines
\usetikzlibrary{shapes, arrows, positioning, automata, calc}
\tikzstyle{UML} = [rectangle split, rectangle split parts=3, draw,text centered,
				   every two node part/.style={align=left, font=\scriptsize}, 
				   every three node part/.style={align=left, font=\scriptsize},
				   font=\small]
\tikzstyle{cas} = [ellipse, draw, text centered, font=\scriptsize]
\tikzstyle{arrow} = [thick,-]
\begin{document}
\title{\textsc{Ken-Ken} \\ \large 1a entrega PROP}
\author{Marc Asenjo i Ponce de León \and
	Arnau Canyadell i Miquel \and
	Joan Marcè i Igual \and
	Iñigo Moreno i Caireta \and
	Esteve Tarragó i Sanchís}

\date{\today}
\maketitle

El nostre grup ha escollit fer el treball de PROP sobre el Ken-Ken. (explicació
sobre el Ken-Ken?). 

En aquest cas l'usuari podrà gestionar Ken-Kens generats per ell o també podrà
deixar que la maquina li generi un. 
Per als Ken-Ken que l'usuari crei el programa comprovarà que hi hagi una solució
possible sinó, el donara com a \emph{no vàlid} i no el deixarà introduir. Els Ken-Ken generats 
pel programa podran tenir uns patrons predefinits tals com la mida del tauler, la forma de les 
regions (hi haurà una llista on es podran escollir les que es volen i les que no) o el nombre 
de nombres ja co\lgem ocats.

En tots dos casos l'usuari podrà intentar resoldre el Ken-Ken desitjat amb
l'assistència del programa que anirà comprovant el compliment de les regles del
joc, en el cas que no es complís alguna norma es mostrarà un avís. Durant el
desenvolupament del joc es tindran en compte les accions de l'usuari per
atorgar-li una puntuació en funció del temps que hagi durat la partida
i també dels errors comesos per l'usuari durant aquesta.

A part també hi haurà un rànquing on quedaran registrades les millors puntuacions,
hi haurà el nom de l'usuari i la puntuació que aquest ha fet.

\vspace{2cm}

\section{Casos d'ús}
%ADD ARROWS
%ADD BLOCKS
\begin{center}
\begin{tikzpicture}[node distance=0.6cm and 0.6cm, auto, align=center]
\node(StartApp){\includegraphics[width=1.3cm]{./Images/stickman}};
\node[cas, right = of StartApp](User){Entrar\\usuari};
\node[cas, right = of User](Ranking){Veure\\rànquing};
\node[cas, above = of Ranking, yshift=5.5cm](Start){Començar\\a jugar};
\node[cas, below = of Ranking](Admin){Veure co\lgem ecció\\de Ken-Kens};
\node[cas, below = of User](Exit){Sortir};

%Començar a jugar
\node[cas, right = of Start](Database){Seleccionar\\tauler de la\\co\lgem ecció};
\node[cas, above = of Database, xshift=1cm](Load){Carregar\\joc};
\node[cas, below = of Database, xshift=-1cm](CreateCPU){Generar i\\jugar};

%Rànquing
\node[cas, right = of Ranking](RRanking){Netejar\\dades};

%Veure col·lecció
\node[cas, right = of Admin](SelectKK){Seleccionar\\Ken-Ken};
\node[cas, below = of Admin](Create){Crear\\Ken-Ken};

\node[cas, right = of CreateCPU, xshift=0.5cm](Play){Jugar};

%Generar i jugar
\node[cas, below = of CreateCPU](SelectSize){Seleccionar\\Mida};
\node[cas, below = of SelectSize](SelectRegions){Seleccionar núm.\\ i mida regions};

%Jugar
\node[cas, right = of Play](Hint){Pista};
\node[cas, above = of Hint](SelectCel){Seleccionar\\ce\lgem a};
\node[cas, below = of Hint](Save){Guardar};
\node[cas, below = of Save](Surrender){Rendir-se};

%Seleccionar cel·la
\node[cas, above right = of SelectCel](ChangeNum){Canviar\\nombre};
\node[cas, right = of SelectCel](DeleteNum){Eliminar\\nombre};
\node[cas, below right = of SelectCel](Notify){Notificació};

%Seleccionar Ken-Ken
\node[cas, right = of SelectKK](EditKK){Editar\\Ken-Ken};
\node[cas, above right = of SelectKK](DeleteKK){Eliminar\\Ken-Ken};
\node[cas, below right = of SelectKK](ViewSolution){Veure\\Solució};

%Crear Ken-Ken
\node[cas, left = of Create](ChooseSize){Triar\\mida};

%Triar mida
\node[cas, below = of ChooseSize, yshift=-4cm](FromZero){De zero};
\node[cas, right = of FromZero, xshift=1cm](CreateRegion){Crear Regió};
\node[cas, above = of CreateRegion](SelectCell){Seleccionar\\ce\lgem a};
\node[cas, above = of SelectCell](FillRandom){Emplenar de\\nombres aleatoris\\correctes};
\node[cas, below = of CreateRegion, xshift=1cm](SelectRegion){Seleccionar\\regió};
\node[cas, below = of SelectRegion, xshift=1cm](ViewSolutionable){Veure si\\té solució};
\node[cas, below = of ViewSolutionable](ViewUniqueSolution){Veure si té\\solució única};
\node[cas, left = of ViewUniqueSolution](SaveNew){Guardar};
\node[cas, below = of FromZero, yshift=-2cm](Solve){Solucionar};

%Seleccionar cel·la
\node[cas, right = of SelectCell, yshift=1cm, xshift=2cm](SetNum){Anotar nombre\\correcte};
\node[cas, above= of SetNum, yshift=-0.5cm](RandomNum){Nombre aleatori\\correcte};

%Crear regió
\node[cas, right = of CreateRegion, yshift=1cm, xshift=2cm](ChangeNum2){Canviar nombre\\ i/o resultat};
\node[cas, below = of ChangeNum2, yshift=0.5cm](SelectCells){Seleccionar\\ce\lgem es};

%Seleccionar regió\\
\node[cas, right = of SelectRegion, xshift=2.2cm](ChangeCells){Modificar\\ce\lgem es};
\node[cas, below = of ChangeCells, yshift=0.5cm](ChangeNumRes){Modificar nombre\\i/o resultat};

\foreach \x/\y in {User/Start, User/Ranking, User/Admin, Start/Database, Start/CreateCPU,
				   Ranking/RRanking, Admin/SelectKK, Admin/Create, Database/Play, 
				   Play/SelectCel, Play/Surrender.north west, Play/Hint, SelectCel/ChangeNum, 
				   SelectCel/DeleteNum, SelectCel/Notify, Play/Save, SelectKK/Play, Start/Load.west, 
				   Load.south east/Play, User/Exit, CreateCPU/SelectSize, SelectKK/ViewSolution,
				   SelectSize/SelectRegions, SelectRegions.north east/Play, StartApp/User,
				   SelectKK/DeleteKK, SelectKK/EditKK, Create/ChooseSize, ChooseSize/FromZero, 
				   FromZero/FillRandom.south west, FromZero/CreateRegion, FromZero/SelectCell, 
				   FromZero/SelectRegion, FromZero/ViewSolutionable, FromZero/ViewUniqueSolution, 
				   FromZero/Solve, SelectCell/RandomNum.south west, SelectCell/SetNum.west, 
				   CreateRegion/ChangeNum2.west, CreateRegion/SelectCells.west, 
				   SelectRegion/ChangeCells, SelectRegion/ChangeNumRes, FromZero/SaveNew} 
{
\draw[thick, ->] (\x) -- (\y);
};

\end{tikzpicture}
\end{center}

\vspace{2cm}

\section{Diagrama de classes}
%ADD ARROWS
\begin{center}
<<<<<<< HEAD
\begin{tikzpicture}[node distance=1cm and 2cm, auto, align=center]
\node[UML](Application){Application \nodepart{two} };
\node[UML, below = of Application](Ranking){Ranking \nodepart{two}};
\node[UML, left = of Ranking, text width=3cm](Player){Player \nodepart{two}
	- Name \\
	- Highscore};
\node[UML, right = of Ranking](Game){Game};
\node[UML, right = of Game](BoardCollection){BoardCollection};
\node[UML, below = of Game, yshift = -1cm](Board){Board};
\node[UML, left = of Board](Cell){Cell};
\node[UML, below = of Cell](Region){Region};
\node[UML, below = of Board](KKBoard){KKBoard};
\node[UML, below = of KKBoard](ConstructionKKBoard){ConstructionKKBoard};
\node[UML, right = of KKBoard](BoardEditor){BoardEditor};
\node[below = of Ranking, yshift=0.5cm](A){};
\node[right = of BoardCollection](B){};

\draw [arrow, -diamond] (Player.north) |- (Application.west)
	node[above left, at start]{*}
	node[above left, at end]{1};
\draw [arrow] (Player.east) |- (Ranking.west) 
	node[above right, at start]{*} 
	node[above left, at end]{1};
\draw [arrow] (Player.south) |- node[below left, at start]{1} (A.center) -| ([xshift=-0.5cm]Game.south)
	node[below left, at end]{1};
\draw [arrow] (Ranking.east) |-  (Game.west) 
	node[above right, at start]{1}
	node[above left, at end]{*};
\draw [arrow, dashed] (Ranking.north) |- (Application.south);
\draw [arrow] (Game.south) |- (Board.north);
\draw [arrow] (Game.north) |- ([yshift=-0.5cm]Application.east);
=======
\begin{tikzpicture}[node distance=1cm and 1.3cm, auto, align=center]
\node[UML](Application){Application \nodepart{two} };
\node[UML, below = of Application](KanKun){KanKun};
\node[UML, below = of KanKun](GameLauncher){GameLauncher};

\node[UML, right = of Application, xshift=0.6cm](DB){DB};
\node[UML, right = of DB](Table){Table};
\node[UML, below = of Table](Data){Data \nodepart{three} Save\\Load};
\node[UML, right = of Table](Ranking){Ranking};
\node[UML, right = of Data](Game){Game \nodepart{two}
	IdGame\\
	Score\\
	Status};
\node[UML, below = of Game](Player){Player \nodepart{two} Nom};

\node[UML, below left = of Data](Board){Board};
\node[UML, below = of Board, yshift = -0.5cm](KKBoard){KKBoard};
\node[UML, right = of Board](Cell){Cell};
\node[UML, right = of KKBoard](Region){Region};

\node[UML, below = of KKBoard](BoardCreator){BoardCreator};
\node[UML, below right = of BoardCreator](BoardHumanEditor){BoardHumanEditor};
\node[UML, below = of BoardCreator](BoardCPUGenerator){BoardCPUGenerator};

\node[left = of GameLauncher, xshift = 1cm](A){};


\draw [arrow, -latex'] (KanKun) -- (Application);
\draw [arrow, dashed, -latex'] (GameLauncher) -- (KanKun);
\draw [arrow, dashed, -latex'] ($(DB.south west)+(0, 0.4)$) -- ++ (-1, 0) |- (GameLauncher.east);
\draw [arrow, dashed, -latex'] (DB) -- (Application)
	node[above right, xshift=1cm]{<<static>>};
\draw [arrow, -diamond] (Table.west) -- (DB.east);
\draw [arrow, -latex'] (Ranking.west) -- (Table.east);
\draw [arrow, -diamond] (Data) -- (Table);
\draw [arrow, -latex'] (Game) -- (Data);
\draw [arrow, -diamond] (Game.north) -- ++ (0, 0) -| (Ranking.south);
\draw [arrow] (Player.west) -- ++ (-0.5, 0) |- (Data.east);
\draw [arrow] (Player) -- (Game)
	node[above right, at start]{1}
	node[below right, at end]{*};
\draw [arrow, -latex'] (Board) |- (Data);
\draw [arrow, -diamond] (Cell) -- (Board);
\draw [arrow, -latex'] (KKBoard) -- (Board);
\draw [arrow, -diamond] (Region) -- (KKBoard);
\draw [arrow] (Region.north) --  ++(0, 0.5) -| (Cell) 
	node[above right, at start, yshift=-0.5cm]{1}
	node[below right, at end]{*};
\draw [arrow, dashed, -latex'] (KKBoard) -- (BoardCreator);
\draw [arrow, -latex'] (BoardHumanEditor) |- (BoardCreator);
\draw [arrow, -latex'] (BoardCPUGenerator) -- (BoardCreator);
\draw [arrow, -latex', dashed] (BoardHumanEditor.south) -- ++ (0, -0.5) -| (A.center) |- (KanKun.west);
\draw [arrow, -latex', dashed] (BoardCPUGenerator) -| (GameLauncher);


%\draw [arrow, -diamond] (Player.north) |- (Application.west)
%	node[above left, at start]{*}
%	node[above left, at end]{1};
%\draw [arrow] (Player.east) |- (Ranking.west) 
%	node[above right, at start]{*} 
%	node[above left, at end]{1};
%\draw [arrow] (Player.south) |- node[below left, at start]{1} (A.center) -| ([xshift=-0.5cm]Game.south)
%	node[below left, at end]{1};
%\draw [arrow] (Ranking.east) |-  (Game.west) 
%	node[above right, at start]{1}
%	node[above left, at end]{*};
%\draw [arrow, dashed] (Ranking.north) |- (Application.south);
%\draw [arrow] (Game.south) |- (Board.north);
%\draw [arrow] (Game.north) |- ([yshift=-0.5cm]Application.east);
>>>>>>> 803ab9a6c18bd24418336135bcffab263d674f36
%\draw [arrow] (Board) -- (KKBoard);
%\draw [arrow] (Cell) -- (Board);
%\draw [arrow] (Cell) -- (Region);
%\draw [arrow] (Region) -- (KKBoard);
%\draw [arrow] (KKBoard) -- (ConstructionKKBoard);
%\draw [arrow] (ConstructionKKBoard.east) |- (BoardEditor.south);
%\draw [arrow] (KKBoard) |- (BoardCollection);
%\draw [arrow] (BoardEditor) -- (BoardCollection);
\draw [arrow, dashed] (BoardCollection) |- ([yshift=1cm]Application);
%\draw [arrow, ]
\end{tikzpicture}
\end{center}
\end{document}
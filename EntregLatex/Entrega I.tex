\documentclass[a4paper,12pt]{article}

\usepackage[catalan]{babel}
\usepackage{fontspec}
\usepackage{fullpage}
\usepackage[a4paper, margin=2cm]{geometry} % To change the margins
\usepackage{graphicx} % Insert images
\usepackage[hidelinks]{hyperref} % Links color
\usepackage{import}
\usepackage{ragged2e}
\usepackage{wrapfig} %To Text wrap
\usepackage{listings} % Add code
\usepackage{tikz}
\usepackage{calc}
\usetikzlibrary{shapes}
\tikzstyle{data} = [rectangle split, rectangle split parts=2, draw,text centered]

\begin{document}
\title{\textsc{Ken-Ken} \\ \large 1a entrega PROP}
\author{Marc Asenjo i Ponce de León \and
	Arnau Canyadell i Miquel \and
	Joan Marcè i Igual \and
	Iñigo Moreno i Caireta \and
	Esteve Tarragó i Sanchís}

\date{\today}
\maketitle

El nostre grup ha escollit fer el treball de PROP sobre el Ken-Ken. (explicació sobre el Ken-Ken?). 

En aquest cas l'usuari podrà gestionar Ken-Kens generats per ell o també podrà deixar que la maquina li generi un. 
Per als Ken-Ken que l'usuari crei el programa comprovarà que hi hagi una solució possible sinó, el donara com a 
\emph{no vàlid} i no el deixarà introduir. 

En tots dos casos l'usuari podrà intentar resoldre el Ken-Ken desitjat amb l'assistència del programa que anirà comprovant
el compliment de les regles del joc, en el cas que no es complís alguna norma es mostrarà un avís.



\begin{tikzpicture}[node distance=1cm, auto, align=center]
	\node[data](Application){Application \nodepart{second} null};
\end{tikzpicture}

\end{document}
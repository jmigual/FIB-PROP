\documentclass[a4paper,12pt]{article}

\usepackage[catalan]{babel}
\usepackage{fontspec}
\usepackage{fullpage}
\usepackage[a4paper, margin=2cm]{geometry} % To change the margins
\usepackage{graphicx} % Insert images
\usepackage[hidelinks]{hyperref} % Links color
\usepackage{import}
\usepackage{ragged2e}
\usepackage{wrapfig} %To Text wrap
\usepackage{listings} % Add code
\usepackage{tikz}
\usepackage{calc}


%Tikz defines
\usetikzlibrary{shapes, arrows, positioning, automata}
\tikzstyle{UML} = [rectangle split, rectangle split parts=3, draw,text centered,
				   every two node part/.style={align=left}, font=\small]
\tikzstyle{cas} = [ellipse, draw, text centered, font=\footnotesize]
\tikzstyle{arrow} = [thick,-]
\begin{document}
\title{\textsc{Ken-Ken} \\ \large 1a entrega PROP}
\author{Marc Asenjo i Ponce de León \and
	Arnau Canyadell i Miquel \and
	Joan Marcè i Igual \and
	Iñigo Moreno i Caireta \and
	Esteve Tarragó i Sanchís}

\date{\today}
\maketitle

El nostre grup ha escollit fer el treball de PROP sobre el Ken-Ken. (explicació
sobre el Ken-Ken?). 

En aquest cas l'usuari podrà gestionar Ken-Kens generats per ell o també podrà
deixar que la maquina li generi un. 
Per als Ken-Ken que l'usuari crei el programa comprovarà que hi hagi una solució
possible sinó, el donara com a \emph{no vàlid} i no el deixarà introduir. Els Ken-Ken generats 
pel programa podran tenir uns patrons predefinits tals com la mida del tauler, la forma de les 
regions (hi haurà una llista on es podran escollir les que es volen i les que no) o el nombre 
de nombres ja co\lgem ocats.

En tots dos casos l'usuari podrà intentar resoldre el Ken-Ken desitjat amb
l'assistència del programa que anirà comprovant el compliment de les regles del
joc, en el cas que no es complís alguna norma es mostrarà un avís. Durant el
desenvolupament del joc es tindran en compte les accions de l'usuari per
atorgar-li una puntuació en funció del temps que hagi durat la partida
i també dels errors comesos per l'usuari durant aquesta.

A part també hi haurà un rànquing on quedaran registrades les millors puntuacions,
hi haurà el nom de l'usuari i la puntuació que aquest ha fet.

\vspace{2cm}

\section{Casos d'ús}
%ADD ARROWS
%ADD BLOCKS
\begin{center}
\begin{tikzpicture}[node distance=1cm and 0.8cm, auto, align=center]
\node(User){\includegraphics[width=1.3cm]{./Images/stickman}};
\node[cas, right = of User](Ranking){Veure\\rànquing};
\node[cas, above = of Ranking](Start){Començar\\a jugar};
\node[cas, below = of Ranking](Admin){Veure co\lgem ecció\\de Ken-Kens};
\node[cas, below = of Admin](Exit){Sortir};

\node[cas, right = of Start](Created){Generat};
\node[cas, above = of Created](Database){Seleccionar\\tauler de la\\co\lgem ecció};

\node[cas, right = of Ranking](RRanking){Netejar\\dades};

\node[cas, right = of Admin](SelectKK){Seleccionar\\Ken-Ken};
\node[cas, below right = of Admin](Create){Crear\\Ken-Ken};

\node[cas, right = of Created](Play){Jugar};

\node[cas, right = of Play](Hint){Pista};
\node[cas, above = of Hint](SelectCel){Seleccionar\\ce\lgem a};
\node[cas, below = of Hint](Save){Guardar};
\node[cas, below = of Save](Surrender){Rendir-se};

\node[cas, above right = of SelectCel](ChangeNum){Canviar\\nombre};
\node[cas, right = of SelectCel](DeleteNum){Eliminar\\nombre};
\node[cas, below right = of SelectCel](Notify){Notificació};

\foreach \x/\y in {User/Start, User/Ranking, User/Admin, Start/Database, Start/Created,
				   Ranking/RRanking, Admin/SelectKK, Admin/Create, Database/Play, 
				   Created/Play, Play/SelectCel, Play/Surrender.north west, Play/Hint, SelectCel/ChangeNum, SelectCel/DeleteNum, SelectCel/Notify, Play/Save,
				   SelectKK/Play} 
{
\draw[thick, ->] (\x) -- (\y);
};

\path[->] (User) edge [bend right] (Exit);
\end{tikzpicture}
\end{center}

\vspace{2cm}

\section{Diagrama de classes}
%ADD ARROWS
\begin{center}
\begin{tikzpicture}[node distance=1cm and 2cm, auto, align=center]
\node[UML](Application){Application \nodepart{two} };
\node[UML, below = of Application](Ranking){Ranking \nodepart{two}};
\node[UML, left = of Ranking, text width=3cm](Player){Player \nodepart{two}
	- Name \\
	- Highscore};
\node[UML, right = of Ranking](Game){Game};
\node[UML, right = of Game](BoardCollection){BoardCollection};
\node[UML, below = of Game, yshift = -1cm](Board){Board};
\node[UML, left = of Board](Cell){Cell};
\node[UML, below = of Cell](Region){Region};
\node[UML, below = of Board](KKBoard){KKBoard};
\node[UML, below = of KKBoard](ConstructionKKBoard){ConstructionKKBoard};
\node[UML, right = of KKBoard](BoardEditor){BoardEditor};
\node[below = of Ranking, yshift=0.5cm](A){};
\node[right = of BoardCollection](B){};

\draw [arrow, -diamond] (Player.north) |- (Application.west)
	node[above left, at start]{*}
	node[above left, at end]{1};
\draw [arrow] (Player.east) |- (Ranking.west) 
	node[above right, at start]{*} 
	node[above left, at end]{1};
\draw [arrow] (Player.south) |- node[below left, at start]{1} (A.center) -| ([xshift=-0.5cm]Game.south)
	node[below left, at end]{1};
\draw [arrow] (Ranking.east) |-  (Game.west) 
	node[above right, at start]{1}
	node[above left, at end]{*};
\draw [arrow, dashed] (Ranking.north) |- (Application.south);
\draw [arrow] (Game.south) |- (Board.north);
\draw [arrow] (Game.north) |- ([yshift=-0.5cm]Application.east);
%\draw [arrow] (Board) -- (KKBoard);
%\draw [arrow] (Cell) -- (Board);
%\draw [arrow] (Cell) -- (Region);
%\draw [arrow] (Region) -- (KKBoard);
%\draw [arrow] (KKBoard) -- (ConstructionKKBoard);
%\draw [arrow] (ConstructionKKBoard.east) |- (BoardEditor.south);
%\draw [arrow] (KKBoard) |- (BoardCollection);
%\draw [arrow] (BoardEditor) -- (BoardCollection);
\draw [arrow, dashed] (BoardCollection) |- ([yshift=1cm]Application);
%\draw [arrow, ]
\end{tikzpicture}
\end{center}
\end{document}